Zelig is highly modular.  You can add methods to Zelig \emph{and}, if
you wish, release your programs as a stand-alone package.  By making
your package compatible with Zelig, you will advertise your package
and help it achieve a widespread distribution.

This chapter assumes that your model is written as a function that
takes a user-defined formula and data set,
and returns a list of output that includes (at the very least) the
estimated parameters and terms that describe the data used to fit the
model.  You should choose a class (either S3 or S4 class) for this
list of output, and provide appropriate methods for generic functions
such as {\tt summary()}, {\tt print()}, {\tt coef()} and {\tt vcov()}.

To add new models to Zelig, you need to provide six R functions,
illustrated in Figure \ref{add}.  Let {\tt mymodel} be a new model
with class {\tt "myclass"}. 

\begin{figure*}[h!]
\caption{Six functions (solid boxes) to implement a new Zelig model}
\label{add}
\begin{center}
\setlength{\unitlength}{0.5mm}
\begin{picture}(160,170)(0,0)
\linethickness{0.75pt}

\put(0,166){Estimate}

\put(70,162){\line(0,-1){42}}
\put(50,162){\dashbox{2}(40,12){{\tt zelig()}}}
\multiput(70,144)(0,-24){2}{\line(1,0){9}}
\put(80,138){\framebox(83,12){(1) {\tt zelig2mymodel()}}}
\put(80,114){\framebox(57,12){(2) {\tt mymodel()}}}

\put(0,96){Interpret}

\put(70,92){\line(0,-1){42}}
\put(50,92){\dashbox{2}(40,12){{\tt sim()}}}
\multiput(70,74)(0,-24){2}{\line(1,0){9}}
\put(80,68){\framebox(83,12){(3) {\tt param.myclass()}}}
\put(80,44){\framebox(69,12){(4) {\tt qi.myclass()}}}

\put(0,26){Plot}

\put(50,0){\framebox(105,12){(6) {\tt plot.zelig.mymodel()}}}

\end{picture}
\end{center}
\end{figure*}

These functions are as follows:  
\begin{enumerate}
\item {\tt zelig2mymodel()} translates {\tt zelig()} arguments into
the arguments for {\tt mymodel()}.
\item {\tt mymodel()} estimates your statistical procedure.
\item {\tt param.myclass()} simulates parameters for your model.
Alternatively, if your model's parameters consist of one vector with a
correspondingly observed variance-covariance matrix, you may write
\emph{two} simple functions to substitute for {\tt param.myclass()}:  
\begin{enumerate}
\item {\tt coef.myclass()} to extract the coefficients from your model
output, and
\item {\tt vcov.myclass()} to extract the variance-covariance matrix
from your model.  
\end{enumerate}
\item {\tt qi.myclass()} calculates expected values, simulates
predicted values, and generates other quantities of interest for your
model (applicable only to models that take explanatory variables).  
\item {\tt plot.zelig.mymodel()} to plot the simulated quantities of
interest from your model.  
\item A {\bf reference manual page} to document the model.
  (See~\Sref{s:format})
\item An optional {\bf demo script} {\tt mymodel.R} which contains commented code for
  the models contained in the example section of your reference manual
  page.
\end{enumerate}

\section{Making the Model Compatible with Zelig}\label{compatible}

You can develop a model, write the model-fitting function, and test it
within the Zelig framework without explicit intervention from the
Zelig team.  (We are, of course, happy to respond to any questions or
suggestions for improvement.)  

Zelig's modularity relies on two R programming conventions:
\begin{enumerate}
\item \emph{\bf wrappers}, which pass arguments from R functions to
  other R functions or to foreign function calls (such as C, C++, or
  Fortran functions); and 
\item \emph{\bf classes}, which tell generic functions how to handle
  objects of a given class.  
\end{enumerate}
Specific methods for R generic functions take the general form: {\tt
  method.class()}, where {\tt method} is the name of the generic
procedure to be performed and {\tt class} is the class of the object.
You may define, for example, {\tt summary.contrib()} to summarize the
output of your model. Note that for S4 classes, the name of generic
functions does not have to be {\tt method.class()} so long as users
can call them via {\tt method()}.

\subsubsection{To Work with {\tt zelig()}}

Zelig has implemented a unique method for incorporating new models
which lets contributors test their models \emph{within} the Zelig
framework \emph{without} any modification of the {\tt zelig()}
function itself.  

Using a wrapper function {\tt zelig2contrib()} (where {\tt contrib} is
the name of your new model), {\tt zelig2contrib()} redefines the inputs to
{\tt zelig()} to work with the inputs you need for your function {\tt
contrib()}.  For example, if you type
\begin{verbatim}
zelig(..., model = "normal.regression")
\end{verbatim}
{\tt zelig()} looks for a {\tt zelig2normal.regression()} wrapper in any
environment (either attached libraries or your workspace).  If the
wrapper exists, then {\tt zelig()} runs the model.  

If you have a pre-existing model, writing a {\tt zelig2contrib()}
function is quite easy.  Let's say that your model is {\tt contrib()},
and takes the following arguments: {\tt formula}, {\tt data}, {\tt
  weights}, and {\tt start}.  The {\tt zelig()} function, in contrast,
only takes the {\tt formula}, {\tt data}, {\tt model}, and {\tt by}
arguments.  You may use the {\tt \dots} to pass additional arguments
from {\tt zelig()} to {\tt zelig2contrib()}, and {\tt <- NULL} to omit
the elements you do not need.  Continuing the Normal regression example
from~\Sref{normal.regression}, let {\tt formula}, {\tt model}, and {\tt data} be
the inputs to {\tt zelig()}, {\tt M} is the number of subsets, and
{\tt \dots} are the additional arguments not defined in the {\tt
  zelig()} call, but passed to {\tt normal.regression()}.

\begin{verbatim}
zelig2normal.regression <- function(formula, data, ...) {
  list(
       .function = "normal.regression", # [1]
       formula = formula,               # [2]
       data = data,                     # [3]
       ...                              # [4]
       )
}
\end{verbatim}

The bracketed numbers above correspond to the comments below: 


\begin{enumerate}

	\item Specify the external method that will be used to fit the data.
  
  \item Specify the formula for the model. Note that if the {\tt data} parameter was omitted, it would be implicitly included by expansion of the dots in step 4. Regardless, it is recommended in general, that {\tt data} and {\tt formula} parameters are always explicitly set.

	\item  Specify the data for the model. Note that if the {\tt data} parameter was omitted, it would be implicitly included by expansion of the dots in step 4. Regardless, it is recommended in general, that {\tt data} and {\tt formula} parameters are always explicitly set.

	\item Implicitly assign any optional parameters that the {\tt normal.regression} method uses. That is, this operation expands the {\tt\dots} and includes them in the call to the {\tt normal.regression} model.
	
\end{enumerate}



\subsubsection{To Work with {\tt setx()}}



In the case of {\tt setx()}, most models will use {\tt setx.default()}, which in turn relies on the generic R function {\tt model.matrix()}.  For this procedure to work, your list of output
must include:  

\begin{itemize}

	\item {\tt terms}, created by {\tt model.frame()}, or manually;

	\item {\tt formula}, the formula object input by the user;

	\item {\tt xlevels}, which define the strata in the explanatory variables; and

	\item {\tt contrasts}, an optional element which defines the type of factor variables used in the explanatory variables.  See {\tt help(contrasts)} for more information.

\end{itemize}


If your model output does not work with {\tt setx.default()}, you must write your own {\tt setx.\emph{contrib}()} function.  For example, models fit to multiply-imputed data sets have output from {\tt zelig()} of class {\tt "MI"}.  The special {\tt setx.MI()} wrapper pre-processes the {\tt zelig()} output object and passes the appropriate arguments to {\tt setx.default()}.  



\subsubsection{Compatibility with {\tt sim()}}

Simulating quantities of interest is an integral part of interpreting model results.  To use the functionality built into the Zelig {\tt sim()} procedure, you need to provide a way to simulate parameters (called a {\tt param()} function), and a method for calculating or drawing quantities of interest from the simulated parameters (called a {\tt qi()} function).


\paragraph{Simulating Parameters}

Whether you choose to use the default method, or write a model-specific method for simulating parameters, these functions require the same three inputs:  

\begin{itemize}
	\item {\tt object}: the estimated model or {\tt zelig()} output.
	
	\item {\tt num}: the number of simulations.
	
	\item {\tt bootstrap**}: either {\tt TRUE} or {\tt FALSE}.  
\end{itemize}

{\tt** }\emph{As of Zelig version 4.0-3, the {\tt bootstrap} parameter is omitted. This feature will be restored as of version 4.1. Sorry for any inconvenience.}


The output from {\tt param()} should be a list containing any of the following slots:

\begin{description}
	\item[simulations] specifies a vector or matrix of simulated parameters, which may be used to simulate particular \emph{quantities of interest} - expected values, predicted values, risk ratios, and average treatment effects.
	
	\item[alpha] specifies the ancillary parameters for this statistical model.
	
	\item[link] specifies the link function, which is included for completely, but is not of great significance.
	
	\item[linkinv] specifies the inverse-link function, and is of particular importance in computing \emph{quantities of interest}.
	
	\item[fam] specifies a {\tt family} object which will implicitly define the {\tt link} and {\tt linkinv} functions.
\end{description}


The general method for simulating parameters is:

\begin{enumerate}

  \item Your model has auxiliary parameters, such as $\sigma$ in the case of the Normal distribution.

  \item Your model performs some sort of correction to the coefficients or the variance-covariance matrix, which cannot be performed in either the {\tt coef.contrib()} or the {\tt vcov.contrib()} functions.

  \item Your model does not rely on asymptotic approximation to the log-likelihood.  For Bayesian Markov-chain monte carlo models, for example, the {\tt param.contrib()} function ({\tt param.MCMCzelig()} in this case) simply extracts the model parameters simulated in the model-fitting function.

\end{enumerate}

Continuing the Normal example, 

\begin{verbatim}
param.normal <- function(obj, num=1000, ...) {
  degrees.freedom <- obj[["df.residual"]]
  sig2 <- summary(obj$result)$dispersion

  list(
       simulations = mvrnorm(n=num, mu=coef(obj), Sigma=vcov(obj)),
       alpha = sqrt(degrees.freedom * sig2 / rchisq(num, degrees.freedom)),
       link = function (x) x,
       linkinv = function (x) x
       )
}

\end{verbatim}


\paragraph{Calculating Quantities of Interest}

All models require a model-specific method for calculating \emph{quantities of interest} from the simulated parameters.  For a model of class {\tt  contrib}, the appropriate {\tt qi()} function is {\tt  qi.contrib()}.  Some suggested \emph{quantities of interest} are:

\begin{itemize}

	\item {\tt ev}:  the expected values, calculated from the analytic solution for the expected value as a function of the systematic component and ancillary parameters.

	\item {\tt pr}:  the predicted values, drawn from a distribution defined by the predicted values.  If R does not have a built-in random generator for your function, you may take a random draw from the uniform distribution and use the inverse CDF method to calculate predicted values.  

	\item {\tt fd}: first differences in the expected value, calculated by subtracting the expected values given the specified {\tt x} from the expected values given {\tt x1}.

	\item {\tt ate.ev}: the average treatment effect calculated using the expected values {\tt ev}.  This is simply {\tt y - ev}, averaged across simulations for each observation.  

	\item {\tt ate.pr}:  the average treatment effect calculated using the predicted values {\tt pr}.  This is simply {\tt y - pr}, averaged across simulations for each observation.

\end{itemize}


The required arguments for the {\tt qi()} function are:  

\begin{description}
	
	\item[{\tt object}] as the zelig output object.

	\item[{\tt x}] as the matrix of explanatory variables (created using {\tt setx()}).

	\item[{\tt x1}] as the optional matrix of alternative values for first differences (also created using {\tt setx()}).  If first differences are inappropriate for your model, you should put in a {\tt warning()} or {\tt stop()} if {\tt x1} is not {\tt NULL}.
	
	\item[{\tt y}] the optional vector or matrix of dependent variables (for calculating average treatment effects).  If average treatment effects are inappropriate for your model, you should put in a {\tt warning()} or {\tt stop()} if conditional prediction has been selected in the {\tt setx()} step.  

	\item[{\tt param}] as the simulated parameters. 
	
\end{itemize}



Continuing the Normal regression example from above, the appropriate {\tt
  qi.normal()} function is as follows:  
\begin{verbatim}
qi.normal <- function(obj, x, x1=NULL, y=NULL, num=1000, param=NULL) {
  # get `num` samples from the underlying distribution
  coef <- coef(param)             # [1]
  alpha <- alpha(param)           # [2]
  inverse.link <- linkinv(param)

  # theta = eta, because inverse of 
  # normal models' link function is
  # the identity
  eta <- inverse.link( coef )     # [3a]
  theta <- matrix(coef %*% t(x), nrow=nrow(coef))

  #
  pr <- matrix(NA, nrow=nrow(theta), ncol=ncol(theta))

  #
  ev <- theta                     # [3b]
  ev1 <- pr1 <- fd <- NA
  
  for (i in 1:nrow(ev))
 
    # [4]
    pr[i,] <- rnorm(ncol(ev), mean = ev[i,], sd = alpha[i])

                                                                                                                                                                                                                  
  # if x1 is not NULL, run more simultations
  # ...

  if (!is.null(x1)) {

    # quantities of interest
    lis1 <- qi(obj, x1, num=num, param=param)

    # pass values over
    ev1 <- lis1[[1]]
    pr1 <- lis1[[3]]

    # compute first differences
    fd <- ev1 - ev
  }

  # return
  list("Expected Values: E(Y|X)" = ev,
       "Expected Values (for X1): E(Y|X1)" = ev1,
       "Predicted Values: Y|X" = pr,
       "Predicted Values: Y|X1" = pr1,
       "First Differences: E(Y|X1) - E(Y|X)" = fd                                                                                                                                               
       )
}

\end{verbatim}

There are five lines of code commented above.  By changing these five lines in the following \emph{four} ways, you can write {\tt qi()} function appropriate to almost any model:  

\begin{enumerate}  
	\item Extract any systematic parameters by substituting the name of your systematic parameter (defined in {\tt describe.mymodel()}).   

	\item Extract any ancillary parameters (defined in {\tt describe.mymodel()}) by substituting their names here.  

	\item Calculate the expected value using the inverse link function and $\eta = X \beta$.  (For the normal model, this is linear.)  You will need to make this change in two places, at Comment [3a] and [3b].

	\item Replace {\tt rnorm()} with a function that takes random draws from the stochastic component of your model.  
\end{enumerate}




\section{Formatting Reference Manual Pages}  \label{s:format}

One of the primary advantages of Zelig is that it fully documents the included models, in contrast to the programming-orientation of R documentation which is organized by function.  Thus, we ask that Zelig contributors provide similar documentation, including the syntax and arguments passed to {\tt zelig()}, the systematic and stochastic components to the model, the quantities of interest, the output values, and further information (including references).  There are several ways to provide this information:  


\begin{itemize}

	\item If you have an existing package documented using the .Rd help format, {\tt help.zelig()} will automatically search R-help in addition to Zelig help.

	\item If you have an existing package documented using on-line HTML files with static URLs (like Zelig or MatchIt), you need to provide a {\tt PACKAGE.url.tab} file which is a two-column table containing the name of the function in the first column and the url in the second.  (Even though the file extension is {\tt .url.tab}, the file should be a tab- or space-delimited text file.)  For example:  

\begin{verbatim}
command       http://gking.harvard.edu/zelig/docs/Main_Commands.html
model         http://gking.harvard.edu/zelig/docs/Specific_Models.html
\end{verbatim}


If you wish to test to see if your {\tt .url.tab} files works, simply place it in your R library/Zelig/data/ directory.  (You do not need to reinstall Zelig to test your {\tt .url.tab} file.)

	\item Preferred method:  You may provide a \LaTeXe\ {\tt .tex} file.  This document uses the book style and supports commands from the following packages:
  {\tt graphicx}, {\tt natbib}, {\tt amsmath}, {\tt amssymb}, {\tt verbatim}, {\tt epsf}, and {\tt html}.  Because model pages are incorporated into this document using {\tt $\backslash$include\{\}}, you should make sure that your document compiles before submitting it.  Please adhere to the following conventions for your model page: 

  \begin{enumerate}

  	\item All mathematical formula should be typeset using the {\tt equation*} and {\tt array}, {\tt eqnarray*}, or {\tt align} environments.  Please avoid {\tt displaymath}.  (It looks funny in html.)

	  \item All commands or R objects should use the {\tt texttt} environment.

  	\item The model begins as a subsection of a larger document, and sections within the model page are of sub-subsection level.

		\item For stylistic consistency, please avoid using the {\tt description} environment.

\end{enumerate}



Each \LaTeX\ model page should include the following elements.  Let {\tt contrib} specify the new model.


\subsubsection*{Help File Template}


\subsubsection{Syntax}


\subsubsection{Examples}
\begin{enumerate}
\item First Example
\item Second Example
\end{enumerate}


\subsubsection{Model}
\begin{itemize}
\item The observation mechanism, if applicable.
\item The stochastic component.
\item The systematic component.
\end{itemize}


\subsubsection{Quantities of Interest}


\begin{itemize}

	\item The expected value of your distribution, including the formula for the expected value as a function of the systemic component and ancillary paramters.  

	\item The predicted value drawn from the distribution defined by the corresponding expected value.  

	\item The first difference in expected values, given when x1 is specified.  

	\item Other quantities of interest.
	
\end{itemize}





\subsubsection{Output Values}
\begin{itemize}
\item From the {\tt zelig()} output stored in {\tt z.out}, you may
  extract:
   \begin{itemize}
   \item 
   \item 
   \end{itemize}
\item From {\tt summary(z.out)}, you may extract: 
   \begin{itemize}
   \item 
   \item 
   \end{itemize}
\item From the {\tt sim()} output stored in {\tt s.out}:
   \begin{itemize}
   \item 
   \item 
   \end{itemize}
\end{itemize}

\subsubsection{Further Information}

\subsubsection{Contributors}
\end{verbatim}
\end{itemize} 



%%% Local Variables: 
%%% mode: latex
%%% TeX-master: "~/zelig/docs/commands/zelig"
%%% End: 
