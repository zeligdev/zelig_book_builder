\documentclass[11pt]{article}
\usepackage{ZeligDoc}
\begin{document}

% TITLE INFORMATION
\title{Making the Model Compatible with Zelig: Writing the \emph{zelig2} Function}
\author{Matthew Owen}
\maketitle


% INTRODUCTION
\section{Introduction}
Developers can develop a model, write the model-fitting function, and test it
within the Zelig framework without explicit intervention from the Zelig team. 
This modularity relies on two R programming conventions:


\begin{enumerate}

	\item {\bf wrappers}, which pass arguments from R functions to other R functions
		or foreign function calls (such as in C, C++, or Fortran).  This step is
		facilitated by - as will be explained in detail in the upcoming chapter -
		the {\tt zelig2} function.
		
	\item {\bf classes}, which tell generic functions how to handle objects of a given
		class.  For a statistical model to be compliant with Zelig, the model-fitting
		function \emph{must} return a classed object.
		
\end{enumerate}

Zelig implements a unique and simple method for incorporating existing statistical
models which lets developers test \emph{within} the Zelig framework \emph{without} any
modification of both their own code or the {\tt zelig} function itself.  The heart of
this procedure is the {\tt zelig2} function, which acts as an interface between the
{\tt zelig} function and the existing statistical model.  That is, the {\tt zelig2}
function maps the user-input from the {\tt zelig} function into input for the existing
statistical model's constructor function.  Specifically, a Zelig model requires:

% !!
%
\begin{enumerate}

	\item An existing statistical model, which is invoked through a function call and
		returns an object

	\item A {\tt zelig2} function which maps user-input from the {\tt zelig} function to
		the existing statistical model
		
	\item A name for the {\tt \bf zelig} model, which can differ from the original name of
		the statistical model.
		
\end{enumerate}


% THE ZELIG2 FUNCTION
\section{The \emph{zelig2} Function}
The following sections explain how to write a {\tt zelig2} function, given an arbitrary
statistical model.  In the illustrative examples, the following conventions are used:



\begin{description}

	\item[model] will refer to the name of the \emph{Zelig} model, not the name of the
		existing model - though these two names are not necessarily different.  If the developer
		names his model ``logit'' then model refers to ``logit''.

	\item[model\_function] will refer to the name of function that produces the existing
		statistical model.  If the developer is writing a wrapper for R's built-in logit function,
		then \emph{model\_function} refers to ``glm''.
		
	\item[zelig2model] will refer to the name of the {\tt zelig2} function.  If the developer
		names his model ``logit'', then \emph{zelig2model} refers to ``zelig2logit''.
		
\end{description}


% WRITING THE ZELIG2 FUNCTION
\subsection{Writing the \emph{zelig2} Function}

The {\tt zelig2} function should follow several specific conventions:

\begin{enumerate}

	\item The {\tt zelig2model} function should be simply named \emph{zelig2model}, where
		\emph{model} is the chosen name for the zelig package

	\item The {\tt zelig2model} function itself should have arguments that list entirety of
		possible inputs to the {\tt model\_function}

	\item The {\tt zelig2model} function should return a list of key-value pairs that represent
		the map from {\tt zelig} input to {\tt model\_function} input

\end{enumerate}


% EXAMPLE USING zelig2logit
\pagebreak
\subsection{Example of a \emph{zelig2} Function}

\begin{verbatim}
zelig2logit <- function(formula, ..., data, weights=NULL)
  list(
       .function = "glm",
        
       formula = formula,
       data    = data,
       weights = weights,
        
       family = binomial(link="logit"),
       model  = FALSE
       )
\end{verbatim}


% EXPLANATION
\subsection{Explanation of \code{zelig2} Return Values}

A {\tt zelig2model} function must always return a list as its return value.

The entries of the returned list have the following format:

\begin{description}

	\item[\code{.function}] specifies, with a character string, {\tt model\_function} used to fit the data
	
	\item[key-value pairs] represent an explicit mapping specified by the developer for the
		parameter that matches ``key''. Value are specified typically in one of two ways:

		\begin{itemize}
		
			\item As a variable based on some information from the user. In the above example, this
				corresponds to the \code{formula}, \code{data}, and \code{weights} keys in the returned
				list. \emph{Note how all the parameters make use of variables specified in the function
				signature}

			\item As a variable that is statically set. This is useful for parameters that do not require
				user-input. In the above example, this corresponds to the \code{family} and \code{model}
				parameters, as their values are specified to specific values regardless of user-input.
				\emph{Note how neither parameter make no use of the parameters in the function signature}
						
		\end{itemize}
	
	\item[an ellipsis (\dots)] specifies that all additional, optional parameters not specified in the signature of the \code{zelig2model\_function} method, will be included in the external method's call, despite not being specifically set.
				
	
\end{description}


\end{document}






















